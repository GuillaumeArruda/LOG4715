%%%%%%%%%%%%%%%%%%%%%%%%%%%%%%%%%%%%%%%%%
% Thin Sectioned Essay
% LaTeX Template
% Version 1.0 (3/8/13)
%
% This template has been downloaded from:
% http://www.LaTeXTemplates.com
%
% Original Author:
% Nicolas Diaz (nsdiaz@uc.cl) with extensive modifications by:
% Vel (vel@latextemplates.com)
%
% License:
% CC BY-NC-SA 3.0 (http://creativecommons.org/licenses/by-nc-sa/3.0/)
%
%%%%%%%%%%%%%%%%%%%%%%%%%%%%%%%%%%%%%%%%%

%----------------------------------------------------------------------------------------
%	PACKAGES AND OTHER DOCUMENT CONFIGURATIONS
%----------------------------------------------------------------------------------------

\documentclass[a4paper, 12pt]{article} % Font size (can be 10pt, 11pt or 12pt) and paper size (remove a4paper for US letter paper)

\usepackage[protrusion=true,expansion=true]{microtype} % Better typography
\usepackage{graphicx} % Required for including pictures
\usepackage[utf8]{inputenc}
\usepackage[margin=1.0in]{geometry}
\usepackage{url}
\usepackage{fancyhdr}
\usepackage{amsmath}
\usepackage{setspace}
\usepackage{enumitem}
\setlength\parindent{0pt} % Removes all indentation from paragraphs

\usepackage[T1]{fontenc} % Required for accented characters
\usepackage{times} % Use the Palatino font

\usepackage{listings}
\usepackage{color}
\lstset{mathescape}

\definecolor{dkgreen}{rgb}{0,0.6,0}
\definecolor{gray}{rgb}{0.5,0.5,0.5}
\definecolor{mauve}{rgb}{0.58,0,0.82}

\lstset{frame=tb,
   language=c++,
   aboveskip=3mm,
   belowskip=3mm,
   showstringspaces=false,
   columns=flexible,
   basicstyle={\small\ttfamily},
   numbers=none,
   numberstyle=\tiny\color{gray},
   keywordstyle=\color{blue},
   commentstyle=\color{dkgreen},
   stringstyle=\color{mauve},
   breaklines=true,
   breakatwhitespace=true
   tabsize=3
}
\linespread{1.00} % Change line spacing here, Palatino benefits from a slight increase by default

\makeatletter
\renewcommand{\@listI}{\itemsep=0pt} % Reduce the space between items in the itemize and enumerate environments and the bibliography

\renewcommand\abstractname{Résumé}
\renewcommand\refname{Références}
\renewcommand\contentsname{Table des matières}
\renewcommand{\maketitle}{ % Customize the title - do not edit title and author name here, see the TITLE block below
\begin{center} % Right align

\vspace*{25pt} % Some vertical space between the title and author name
{\LARGE\@title} % Increase the font size of the title

\vspace{125pt} % Some vertical space between the title and author name

{\large\@author} % Author name

\vspace{125pt} % Some vertical space between the author block and abstract
Dans le cadre du cours
\\LOG4715 - Conception de jeux vidéo
\vspace{125pt} % Some vertical space between the author block and abstract
\\\@date % Date
\vspace{125pt} % Some vertical space between the author block and abstract

\end{center}
}

%----------------------------------------------------------------------------------------
%	TITLE
%----------------------------------------------------------------------------------------

\title{TP3: Conception et implantation d'un niveau de jeu} 

\author{\textsc{Guillaume Arruda 1635805 \\ Raphael Lapierre 1644671 \\ Éric Morissette 1631103} % Author
\vspace{10pt}
\\{\textit{École polytechnique de Montréal}}} % Institution

\date{7 Décembre 2015} % Date

%----------------------------------------------------------------------------------------

\begin{document}

\thispagestyle{empty}
\clearpage\maketitle % Print the title section
\pagebreak[4]
%----------------------------------------------------------------------------------------
%	En tête et pieds de page 
%----------------------------------------------------------------------------------------

\setlength{\headheight}{15.0pt}
\pagestyle{fancy}
\fancyhead[L]{LOG4715}
\fancyhead[C]{}
\fancyhead[R]{TP3 : Conception et implantation d'un niveau de jeu}
\fancyfoot[C]{\textbf{page \thepage}}

%----------------------------------------------------------------------------------------
%	ESSAY BODY
%----------------------------------------------------------------------------------------
\section*{Entrepôt}
L'entrepôt de la solution est disponible à l'adresse suivante: https://github.com/GuillaumeArruda/LOG4715

\section*{Explication}
\subsection*{Contrôle}
Les contrôles sont similaires à la remise du projet 2. WASD contrôlent le déplacement de la voiture au sol. Espace contrôle le saut,
Shift la nitro, E et Q la rotation dans les airs et la souris les différents types de projectiles.
\subsection*{Thématique}
Notre équipe d’artistes a travaillé jour et nuit pour créer de magnifiques textures à la main. Ces textures créées à l’aide de technologie de pointe font ressortir la nostalgie du joueur en lui rappelant les heures perdues a innover sur Paint plutôt que de travailler lors de son enfance. Ces textures viennent aussi s’inscrire dans la thématique Loco de notre jeu. Nous avons opté pour un jeu rapide oè les dérapages et accidents sont courant.
\subsection*{Nouveau tracé}
Le nouveau tracé de la piste comporte plusieurs aspects pouvant intéressants au joueur: Il y a des pentes montantes et descendantes, il y a des sauts, des virages secs et larges,	des endroits où la piste se sépare en deux, donnant un choix au joueur, ainsi que plusieurs raccourcis comportants leurs avantages et risques.
\subsection*{Guidage}
Pour le guidage du joueur, nous avons opté pour quelques façons simples de diriger le joueur.	Premièrement, nous avons ajoutés des grosses flèches sur le sol afin de pointer la direction dans laquelle il faut aller. Deuxièmement, nous avons augmenté le contraste et les dimensions des murs afin de les rendre plus faciles à voir pour le joueur. Troisièmement, l'utilisation d'objets en mouvement permet de dire au joueur où se trouvent des raccourcis risqués.
\subsection*{Raccourci}
Nous avons ajoutés plusieurs raccourcis dans le tracé de la piste afin de rendre la course plus intéressante au joueur humain étant capable de comprendre les différentes possibilités s'offrant à lui. Nous avons un saut passant au travers d'un trou dans un mur afin de sauver un léger détour, un virage qui peut être coupé si l'on passe en dessous de plusieurs objets tournants sur eux mêmes, qui bloquent en partie le chemin, 	un passage caché par un objet en mouvement qui permet de se rendre au fil d'arrivé plus rapidement.
\subsection*{Éléments de piste en mouvement}
Nous avons ajoutés quelques objets en mouvement afin de bloquer les mouvements du joueur, sans affecter les joueurs artificiels. Ces blocs en mouvement bloquent certains raccourcis pouvant être pris par le joueur humain.
\subsection*{Récompenses}
Les mêmes récompenses qu'au TP2 ont étées ajoutées au nouveau tracé afin de donner plus d'éléments de choix et de pouvoir au joueur.
\subsection*{Replacement}
L'algorithme de replacement à été très légèrement modifié afin de mieux être adapté au nouveau tracé de la piste.
\vspace{10pt}
%----------------------------------------------------------------------------------------
\end{document}
